\documentclass[12pt,a4paper]{article}
\usepackage[utf8]{inputenc}
\usepackage[english]{babel}
\usepackage{amsmath}
\usepackage{amsfonts}
\usepackage{amssymb}
\usepackage{graphicx}
\usepackage{booktabs}
\usepackage{array}
\usepackage{longtable}
\usepackage{listings}
\usepackage{xcolor}
\usepackage{geometry}
\usepackage{fancyhdr}
\usepackage{hyperref}

\geometry{margin=1in}
\pagestyle{fancy}
\fancyhf{}
\rhead{A* Algorithm in Rush Hour Solver}
\lfoot{\today}
\rfoot{\thepage}

% Code styling
\lstset{
    language=Python,
    basicstyle=\ttfamily\small,
    keywordstyle=\color{blue},
    commentstyle=\color{green},
    stringstyle=\color{red},
    numbers=left,
    numberstyle=\tiny,
    frame=single,
    breaklines=true,
    showstringspaces=false
}

\title{A* Algorithm in Rush Hour Solver}
\author{AI Project Documentation}
\date{\today}

\begin{document}

\maketitle
\tableofcontents
\newpage

\section{Theory of A* Algorithm}

\subsection{Introduction}

The A* (A-star) algorithm is an informed search algorithm that combines the best features of uniform-cost search and greedy best-first search. It uses both the actual cost from the start and a heuristic estimate to the goal to find optimal solutions efficiently.

\subsection{Core Concept}

A* uses an evaluation function to guide its search:
\begin{equation}
f(n) = g(n) + h(n)
\end{equation}

Where:
\begin{itemize}
    \item $f(n)$ = estimated total cost through node $n$
    \item $g(n)$ = actual cost from start to node $n$
    \item $h(n)$ = heuristic estimate from node $n$ to goal
\end{itemize}

\subsection{Key Properties}

\begin{table}[h!]
\centering
\begin{tabular}{|l|p{6cm}|}
\hline
\textbf{Property} & \textbf{Description} \\
\hline
Optimal & Finds shortest path if heuristic is admissible \\
\hline
Complete & Always finds solution if one exists \\
\hline
Efficient & Uses heuristic to guide search toward goal \\
\hline
Admissible Heuristic & Never overestimates true cost: $h(n) \leq h^*(n)$ \\
\hline
\end{tabular}
\caption{A* Algorithm Properties}
\end{table}

\subsection{Basic Algorithm}

\begin{enumerate}
    \item Start with initial node in open set
    \item While open set is not empty:
    \begin{enumerate}
        \item Select node with lowest $f(n)$ value
        \item If goal reached, return solution path
        \item Move node to closed set
        \item Add valid neighbors to open set
    \end{enumerate}
    \item Return failure if no solution found
\end{enumerate}

\section{Detailed Description of A* Algorithm}

\begin{table}[h!]
\centering
\begin{tabular}{|l|l|}
\hline
\textbf{Information} & \textbf{Details} \\
\hline
Algorithm Name & A* Search (A-Star Search) \\
\hline
Algorithm Type & Informed Search Algorithm \\
\hline
Time Complexity & $O(b^d)$ in worst case, but usually much more efficient \\
\hline
Space Complexity & $O(b^d)$ \\
\hline
\end{tabular}
\caption{A* Algorithm Overview}
\end{table}

\section{Input}

\begin{table}[h!]
\centering
\begin{tabular}{|l|l|p{6cm}|}
\hline
\textbf{Parameter} & \textbf{Data Type} & \textbf{Description} \\
\hline
\texttt{board} & Board object & Rush Hour game board containing grid size and goal position information \\
\hline
\texttt{initial\_state} & State object & Initial state of the puzzle, containing positions of all vehicles \\
\hline
\end{tabular}
\caption{Input Parameters}
\end{table}

\subsection{Input Details}
\begin{itemize}
    \item \textbf{Board}: Contains information about:
    \begin{itemize}
        \item Grid dimensions (width $\times$ height)
        \item Goal position (goal\_pos) that the red car needs to reach
        \item Movement space constraints
    \end{itemize}
    
    \item \textbf{Initial State}: Contains:
    \begin{itemize}
        \item Dictionary of all vehicles with IDs and position information
        \item Red car (ID = 'X') is the target vehicle to reach the goal
    \end{itemize}
\end{itemize}

\section{Output}

\begin{table}[h!]
\centering
\begin{tabular}{|l|l|p{6cm}|}
\hline
\textbf{Result} & \textbf{Data Type} & \textbf{Description} \\
\hline
Success & List[State] & List of states from initial state to goal state \\
\hline
Failure & None & No solution found \\
\hline
\end{tabular}
\caption{Output Specifications}
\end{table}

\subsection{Output Details}
\textbf{List[State]}: Each State in the list represents one move step
\begin{itemize}
    \item State[0]: Initial state
    \item State[1], State[2], ...: Intermediate move steps
    \item State[n-1]: Goal state (red car has escaped)
\end{itemize}

\section{Detailed Algorithm Operation}

\subsection{Initialization}

\begin{lstlisting}[caption=Initialization Code]
initial_node = Node(self.initial_state, path_cost=0)
frontier = [(self._blocking_heuristic(initial_node.state), initial_node)]
explored = {initial_node.state: 0}
\end{lstlisting}

\begin{table}[h!]
\centering
\begin{tabular}{|l|p{8cm}|}
\hline
\textbf{Component} & \textbf{Description} \\
\hline
\texttt{initial\_node} & First node with initial state and cost = 0 \\
\hline
\texttt{frontier} & Priority queue containing unexplored nodes, sorted by $f(n) = g(n) + h(n)$ \\
\hline
\texttt{explored} & Dictionary storing explored states and their best costs \\
\hline
\end{tabular}
\caption{Initialization Components}
\end{table}

\subsection{Main Loop}

\begin{table}[h!]
\centering
\begin{tabular}{|c|p{4cm}|p{6cm}|}
\hline
\textbf{Step} & \textbf{Action} & \textbf{Details} \\
\hline
1 & Get node with lowest priority & \texttt{current\_node = heapq.heappop(frontier)} \\
\hline
2 & Check termination condition & If red car reached goal position $\rightarrow$ return path \\
\hline
3 & Generate successor states & All valid moves from current state \\
\hline
4 & Calculate cost & $g(n) = \text{path\_cost} + \text{move\_cost}$ \\
\hline
5 & Calculate heuristic & $h(n) = \text{blocking\_heuristic(state)}$ \\
\hline
6 & Calculate priority & $f(n) = g(n) + h(n)$ \\
\hline
7 & Update frontier and explored & Add new node if unexplored or has better cost \\
\hline
\end{tabular}
\caption{Main Loop Steps}
\end{table}

\subsection{Heuristic Function (Blocking Heuristic)}

\begin{lstlisting}[caption=Blocking Heuristic Implementation]
def _blocking_heuristic(self, state):
    red_car = state.vehicles['X']
    grid = state._create_grid(self.board)
    blocking_cars = set()
    
    for i in range(red_car.x + red_car.length, self.board.width):
        if grid[red_car.y][i] != '.':
            blocking_cars.add(grid[red_car.y][i])
    
    return len(blocking_cars)
\end{lstlisting}

\begin{table}[h!]
\centering
\begin{tabular}{|l|p{8cm}|}
\hline
\textbf{Component} & \textbf{Description} \\
\hline
Purpose & Estimate minimum steps needed for red car to escape \\
\hline
Calculation & Count distinct vehicles blocking red car's path to exit \\
\hline
Properties & Admissible (never overestimates) and Consistent \\
\hline
Range & From 0 (no blocking cars) to $n$ ($n$ blocking cars) \\
\hline
\end{tabular}
\caption{Heuristic Function Properties}
\end{table}

\subsection{Cost Calculation}

\begin{table}[h!]
\centering
\begin{tabular}{|l|l|p{5cm}|}
\hline
\textbf{Cost Type} & \textbf{Formula} & \textbf{Meaning} \\
\hline
$g(n)$ & \texttt{current\_cost + vehicle.length} & Actual cost from initial state \\
\hline
$h(n)$ & \texttt{blocking\_heuristic(state)} & Estimated cost to goal \\
\hline
$f(n)$ & $g(n) + h(n)$ & Total estimated cost \\
\hline
\end{tabular}
\caption{Cost Functions}
\end{table}

\subsection{Termination Conditions}

\begin{table}[h!]
\centering
\begin{tabular}{|l|l|}
\hline
\textbf{Condition} & \textbf{Action} \\
\hline
Goal found & \texttt{return self.\_reconstruct\_path(current\_node)} \\
\hline
Empty frontier & \texttt{return None} (no solution exists) \\
\hline
\end{tabular}
\caption{Termination Conditions}
\end{table}

\section{Operation Example}

\subsection{Initial State}
\begin{verbatim}
+---+---+---+---+---+---+
| A | A |   |   |   |   |
+---+---+---+---+---+---+
| X | X | B | B |   |   |
+---+---+---+---+---+---+
|   |   | C |   |   |   |
+---+---+---+---+---+---+
\end{verbatim}

\subsection{Calculation Steps}
\begin{enumerate}
    \item $h(\text{initial}) = 2$ (cars B and C block car X)
    \item $g(\text{initial}) = 0$ (initial cost)
    \item $f(\text{initial}) = 0 + 2 = 2$
\end{enumerate}

\subsection{After moving car A down}
\begin{verbatim}
+---+---+---+---+---+---+
|   |   |   |   |   |   |
+---+---+---+---+---+---+
| X | X | B | B |   |   |
+---+---+---+---+---+---+
| A | A | C |   |   |   |
+---+---+---+---+---+---+
\end{verbatim}

\begin{itemize}
    \item $h(\text{new}) = 2$ (cars B and C still blocking)
    \item $g(\text{new}) = 2$ (cost of moving car A with length 2)
    \item $f(\text{new}) = 2 + 2 = 4$
\end{itemize}

\section{Advantages of A* in Rush Hour}

\begin{table}[h!]
\centering
\begin{tabular}{|l|p{8cm}|}
\hline
\textbf{Advantage} & \textbf{Description} \\
\hline
Optimal & Always finds optimal solution if heuristic is admissible \\
\hline
Efficient & Good heuristic reduces number of nodes to explore \\
\hline
Complete & Always finds solution if one exists \\
\hline
Intelligent & Prioritizes exploring most promising paths \\
\hline
\end{tabular}
\caption{A* Algorithm Advantages}
\end{table}

\section{Comparison with Other Algorithms}

\begin{table}[h!]
\centering
\begin{tabular}{|l|c|c|c|c|}
\hline
\textbf{Algorithm} & \textbf{Optimal} & \textbf{Complete} & \textbf{Efficient} & \textbf{Memory} \\
\hline
BFS & \checkmark & \checkmark & $\times$ & High \\
\hline
DFS & $\times$ & $\times$ & $\times$ & Low \\
\hline
UCS & \checkmark & \checkmark & $\triangle$ & High \\
\hline
A* & \checkmark & \checkmark & \checkmark & High \\
\hline
\end{tabular}
\caption{Algorithm Comparison}
\end{table}

\section{Conclusion}

The A* algorithm provides an optimal and efficient solution for the Rush Hour puzzle by combining the benefits of uniform-cost search with an admissible heuristic function. The blocking heuristic effectively guides the search towards promising states while maintaining optimality guarantees.

The implementation demonstrates the power of informed search algorithms in constraint satisfaction problems, where domain-specific knowledge can significantly improve performance compared to uninformed search strategies.

\end{document}
